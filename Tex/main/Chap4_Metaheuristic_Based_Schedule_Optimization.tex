\chapter{负载划分及调度优化算法的设计与实现}\label{chap:Metaheuristic_Based_Schedule_Optimization} 

\section{基于决策树回归模型的性能量化模型训练流程}\label{sec:decision_tree_based_quant_regression_procedure}
\section{基于元启发算法的参数优化基本流程}\label{sec:metaheuristic_optimization_procedure}

\section{基于UPMEM系统的性能量化回归学习模块设计与实现}\label{sec:UPMEM_SKMD_regression_model}
    \subsection{运行时性能探针ChronoTrigger}\label{subsec:runtime_perf_probe_impl}
    \subsection{基于决策树的性能量化模型}\label{subsec:decision_tree_based_model_impl}
    \subsection{性能模型预热与检查点机制}\label{subsec:warmup_and_checkpoint_impl}
    
\section{面向负载性能量化回归模型的元启发调度优化模块实现}\label{sec:metaheuristic_optimizer_facing_SKMD_regression_model}
  %方法部分,讲述“切分+按比例派发+同时运行”形而上学的方法论、以及其成功的理论依据
  %--------------- 此部分采用“分总”结构叙事 -------------
    \subsection{调度映射算法}\label{subsec:schedule_vec_mapping_algorithm_design}
    \subsection{可变权重调度价值函数}\label{subsec:schedule_evaluation_func_design}
    \subsection{元启发算法实现及通用优化器接口设计}\label{subsec:metaheuristic_algorithm_implement_and_general_interface}

\section{本章小结}\label{sec:chap4_summary}